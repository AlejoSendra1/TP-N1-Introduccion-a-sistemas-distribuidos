\section{Conclusiones}

El desarrollo de este trabajo práctico permitió implementar y evaluar un sistema de transferencia de archivos confiable sobre UDP, explorando las diferencias entre los protocolos Stop \& Wait y Selective Repeat. Mientras que el primero ofreció simplicidad a costa de eficiencia y gran desempeño para paquetes pequeños, el segundo demostró un mejor desempeño en la medida que se aumentaba el porcentage de pérdidas de paquetes gracias a su ventana deslizante y retransmisiones selectivas. El diseño modular de la solución y la definición de un protocolo de aplicación propio facilitaron la gestión de errores, el soporte de concurrencia y el cierre seguro de sesiones. En conjunto, la experiencia brindó una comprensión más profunda de los mecanismos de confiabilidad en redes distribuidas y su relevancia práctica para sistemas de comunicación robustos.